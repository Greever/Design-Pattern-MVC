\chapter*{Abk�rzungsverzeichnis}
\addcontentsline{toc}{chapter}{Abk�rzungsverzeichnis}

\begin{acronym}
\acro{AJAX}{Asynchronous JavaScript and XML}
\acro{API}{Application Programming Interface}
\acro{ASP.NET}{Active Server Pages .NET}
\acro{CIL}{Common Intermediate Language}
\acro{CLI}{Common Language Runtime}
\acro{CLR}{Common Language Runtime}
\acro{CSDL}{Conceptual Schema Definition Language}
\acro{CSS}{Cascad Style Sheet}
\acro{FCL}{Framework Class Library}
\acro{HTML}{Hypertext Markup Language}
\acro{HTTP}{Hypertext Transfer Protocol}
\acro{IIS}{Internet Information Server}
\acro{JIT}{Just-In-Time-Compiler}
\acro{JSON}{JavaScript Object Notation}
\acro{LINQ}{Language Integrated Query}
\acro{MSL}{Mapping Specification Language}
\acro{OO}{Objekt Orientierung}
\acro{RE}{Requirements Engineering}
\acro{REST}{Representational State Transfer}
\acro{RPC}{Remote Procedure Call}
\acro{SOA}{Service Orientierte Architektur}
\acro{SOAP}{Simple Object Access Protocol}
\acro{SSDL}{Store Schema Definition Language}
\acro{TCP}{Transmission Control Protocol}
\acro{TFS}{Team Foundation Server}
\acro{UDDI}{Universal Description Discovery and Integration}
\acro{WCF}{Windows Communication Foundation}
\acro{WPF}{Windows Presentation Foundation}
\acro{WS}{Web Service}
\acro{WSDL}{Web Service Description Language}
\acro{XML}{Extensible Markup Language}
\acro{XML}{Extensible Application Markup Language}
\end{acronym}